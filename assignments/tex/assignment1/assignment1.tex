\documentclass[a4paper,10pt]{article}
\usepackage[margin=2cm]{geometry}
\usepackage[utf8]{inputenc}



\newcommand{\trnsp}{\mathsf{T}}
\usepackage[round]{natbib}
\renewcommand{\bibname}{References}
\bibliographystyle{abbrvnat}

\usepackage{amsmath, amsfonts}
\usepackage{graphicx}
\usepackage{subfig}
\usepackage{float}
\usepackage{hyperref}
\hypersetup{colorlinks=true, urlcolor=blue, citecolor=magenta}


\makeatletter
\@addtoreset{exercise}{section}
\makeatother  
\newcounter{exercise}
\newtheorem{exercise}{Exercise}
\renewcommand{\theexercise}{\arabic{exercise}}
\newsavebox{\mybox}
\newenvironment{note}
{
\begin{center}
\begin{lrbox}{\mybox}
\begin{minipage}{42em}}
{\end{minipage}
\end{lrbox}\fbox{\usebox{\mybox}}
\end{center}}


\title{Seminars on Overparametrized Machine Learning: \\Hand-in assignment 1}
\author{Antônio H. Ribeiro, Dave Zachariah, Per Mattsson}

\begin{document}

\maketitle
\begin{center}
	\large \textbf{Due: 30th  of September 2021, 23:59}
\end{center}

\textit{
The items in the assignment can be implemented in the programming language of choice.
Nonetheless, we recommend the usage of Python as a programming language, since we might include suggestions of functions and code snippets in the exercise description.}



\section*{Double-descent in random Feature regression}


Follow the instructions to reproduce the double-descent behaviour in a regression problem. The setup resembles that used in Figure 2 in~\citep{belkin_reconciling_2019}: you will use random Fourier features and the minimum-norm solution. For simplicity, however, it will be used a smaller dataset and one that deals with regression (rather than classification).

\paragraph{Dataset.} You will use the Boston house price dataset. The dataset is available in:\\
    \href{https://archive.ics.uci.edu/ml/machine-learning-databases/housing/}{https://archive.ics.uci.edu/ml/machine-learning-databases/housing/}. 
    
    \begin{note}
    \textbf{Note:} If you are using Python, scikit-learn library does have a function for loading the dataset, see \href{https://scikit-learn.org/stable/modules/generated/sklearn.datasets.load_boston.html}{sklearn.datasets.load\_boston}.
    \end{note}
    
    Divide the dataset into 60\% / 40\% splits. Using the 60\% split for training and the 40\% split for testing the model.
    
\paragraph{Model.} Consider a linear-in-the-parameter model for predicting the output from the input:
    \begin{equation}
        \label{eq:nonlin-map}
        \hat{y}_j = f(x_j) = \sum_{i=1}^m \theta_i \phi_i(x_j). 
    \end{equation}
    where $\phi_i$, $i=1, \cdots, m$ are nonlinear transformations that map the input to  the feature space. As in~\citet{belkin_reconciling_2019} Figure 2, you will use random Fourier features (see a more detailed description next).
    
    
    Given the training dataset $\{(x_j, y_j), j=1, \cdots, n\}$, the model is estimated by finding the values $\theta_i$ that minimize
    \begin{equation}
        \label{eq:estimation}
        \frac{1}{T}\sum_{t = 1}^T\left( y_t - \sum_{i=1}^m \theta_i \phi_i(x_j)\right)^2.
    \end{equation}
    Or, equivalently, in matrix form, by finding the vector $\theta \in \mathbb{R}^m$ that minimizes
    \begin{equation}
         \label{eq:estimation-matrix}
        \frac{1}{T}\|y - \Phi \theta\|^2,  
    \end{equation}
    where $\Phi \in\mathbb{R}^{n\times m}$ is the matrix containing  $\phi_j(x_i)$ at position $(i, j)$ and $y \in \mathbb{R}^T$ is the vector of outputs. Indeed, finding  the  optimal  parameter  here  is  an  ordinary  least-squares problem and its analytical solution is
    \begin{equation}
    \label{eq:ls-sol}
    \hat{\theta} = (\Phi^\trnsp \Phi)^{+}\Phi^\trnsp y,
    \end{equation}

    In the overparametrized regime, there are multiple solutions, and it is possible to prove that, in this case,  using the above solution yields the minimum $\ell_2$-norm solution to the problem (as used by \citet{belkin_reconciling_2019}), i.e.:
    \begin{equation}
    \label{eq:min-norm-solution}
    \hat{\theta} = \text{arg}\min_\theta \|\theta\|_2 \quad \text{subject to}\quad\Phi\theta = y,
    \end{equation}
    where $(\Phi^\trnsp \Phi)^{+}$ denotes the Moore-Penrose pseudo-inverse of $\Phi^\trnsp \Phi$.  
    
    \begin{note}
    \textbf{Note:} \href{https://docs.scipy.org/doc/scipy/reference/generated/scipy.linalg.lstsq.html}{scipy.linalg.lstsq} does yield the desired behaviour both in the underparametrized and overparametrized region.
    \end{note}
    
    
    \paragraph{Random Fourier features} Use the feature map introduced by 
    \citet{rahimi_random_2008} to approximate the reproducing kernel Hilbert space (RKHS) defined by the Gaussian kernel ${K(x, x') = \exp(-\gamma \|x - x'\|^2)}$. More precisely, the features are generated as
    \begin{equation}
        \phi_i(x) = \sqrt{\frac{2}{m}}\cos(w_i^\trnsp x + b_i),
    \end{equation}
    where $w_i\in\mathbb{R}^n$ is a vector with each element sampled independently from $\mathcal{N}(0, 2\gamma)$ and $b_i\in\mathbb{R}$ is sampled from a uniform distribution $\mathcal{U}[0, 2\pi)$. 
    
    \begin{note}
    \textbf{Note:} There is a implementation of random Fourier features in scikit-learn that you can use:
    \href{https://scikit-learn.org/stable/modules/generated/sklearn.kernel_approximation.RBFSampler.html}{sklearn.kernel\_approximation.RBFSampler}.
    \end{note}
    
    \begin{note}
    You might notice that the formulation above is slightly different than the one presented in~\citet{belkin_reconciling_2019} for RFF. They are equivalent tough, and the one above has the advantage of do not be expressed using complex numbers.
    \end{note}
    

    

\section*{Exercise}

Let $n$ be the number of training points, solve the above problem for $m$ ranging from $0.1n$ to $10 n$ (generate at least 100 configurations in this range using logspace). Plot as a function of the number of features $m$:
\begin{enumerate}
    \item \textbf{Train} mean square error; 
    \item \textbf{Test} mean square error;
    \item  The \textbf{parameter $\ell_2$ norm} $\|\theta\|_2$.
\end{enumerate}

   \begin{note}
    Some tips for the plots to look nice:
    \begin{itemize}
        \item Close to the interpolation point the test error takes very
        large values. We suggest to manually set the y-limits in the plot, so the region of interest is highlighted. The same also applies to the parameter norm.
        \item Using logscale in the x-axis might also make the plot more clear.
        \item Maybe add a vertical line in the threshold $m=n$ to show where the interpolation threshold is.
    \end{itemize}
    \end{note}

\section*{The Submission}
Your submission should have a single page of content (a4paper, fontsize=10pt, margin=2cm, both single and double column are acceptable...).  Include your name, the plots, a short description of experiment parameters that you used (i.e., which $\gamma$) and a paragraph of discussion/conclusion. You can assume that whoever will read your report has both read paper from~\citep{belkin_reconciling_2019} and the entire description above, so there is no need for repeating it... 

All requested plots should have proper figure captions, legends, and axis labels. You should submit two files, one pdf-file with the report and a standalone script (or jupyter notebook) that can be used to run the code and generate the plots (Write as comments  the packages/libraries versions and additinal requirements as comments in the top of the script).

\begin{thebibliography}{2}
\providecommand{\natexlab}[1]{#1}
\providecommand{\url}[1]{\texttt{#1}}
\expandafter\ifx\csname urlstyle\endcsname\relax
  \providecommand{\doi}[1]{doi: #1}\else
  \providecommand{\doi}{doi: \begingroup \urlstyle{rm}\Url}\fi

\bibitem[Belkin et~al.(2019)Belkin, Hsu, Ma, and
  Mandal]{belkin_reconciling_2019}
M.~Belkin, D.~Hsu, S.~Ma, and S.~Mandal.
\newblock Reconciling modern machine-learning practice and the classical
  bias\textendash variance trade-off.
\newblock \emph{Proceedings of the National Academy of Sciences}, 116\penalty0
  (32):\penalty0 15849--15854, Aug. 2019.
\newblock ISSN 0027-8424, 1091-6490.
\newblock \doi{10.1073/pnas.1903070116}.

\bibitem[Rahimi and Recht(2008)]{rahimi_random_2008}
A.~Rahimi and B.~Recht.
\newblock Random {{Features}} for {{Large}}-{{Scale Kernel Machines}}.
\newblock In \emph{Advances in {{Neural Information Processing Systems}} 20},
  pages 1177--1184. 2008.

\end{thebibliography}

\end{document}
